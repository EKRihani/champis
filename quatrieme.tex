%\documentclass[12pt,a4paper]{article} % POUR TEST
%\usepackage[french]{babel} % POUR TEST
%\usepackage[utf8]{inputenc} % POUR TEST
%\usepackage[T1]{fontenc} % POUR TEST
%\usepackage{vmargin} % POUR TEST
%\setmarginsrb{2.5cm}{1.5cm}{2.5cm}{2cm}{0cm}{0cm}{0cm}{0cm} % POUR TEST
%\begin{document} % POUR TEST
\cleardoublepage 
~
\newpage
\pagestyle{empty}
{\sffamily
\vspace{22mm}
\begin{center}
Université de Lille \\
FACULTE DE PHARMACIE DE LILLE \\
\textbf{DIPLOME D'ETAT DE DOCTEUR EN PHARMACIE}\\
Ann\'{e}e Universitaire 2022/2023
\end{center}
\vspace{15mm}
\noindent{\textbf{Nom :} RIHANI\\
\textbf{Prénom :} Emir Kaïs}

\vspace{10mm}
\noindent{\textbf{Titre de la thèse :} Application de modèles d'apprentissage machine à la classification des macromycètes}

\vspace{10mm}
\noindent{\textbf{Mots-clés :} intelligence artificielle, apprentissage machine, \emph{machine learning}, classification, R, champignons, mycologie}

\vspace{13mm}
\noindent{\rule{16cm}{0.8pt}}

\vspace{4mm}
\noindent{\textbf{Résumé :} Cette étude propose de soumettre un lot synthétique de macromycètes à différents types de modèles d'apprentissage machine supervisé (analyse discriminante linéaire, arbres décisionnels, forêts aléatoires) et d'évaluer leurs performances respectives dans des tâches de classification par critères de comestibilité, par familles, puis par espèces.
}

\vspace{21mm}     % ESPACE A AJUSTER SELON TEXTE CI-DESSUS
\noindent{\rule{16cm}{0.8pt}}

\vspace{4mm}
\noindent{\textbf{\underline{\smash{Membres du jury}} :} \\
~\\
\textbf{Président :} \\ 
Pr LEMDANI Mohammed, PU en Biomathématiques, Faculté de Pharmacie de Lille \\
~\\
\textbf{Directeur, conseiller de thèse :} \\
Dr HAMONIER Julien, MCU en Biomathématiques, Faculté de Pharmacie de Lille \\
~\\
\textbf{Assesseur :} \\
Dr WELTI Stéphane, MCU en Sciences Végétales et Fongiques, Faculté de Pharmacie de Lille \\
~\\
\textbf{Membre extérieur :} \\
Dr MOUSSET Caroline, Pharmacien-Ingénieur, Responsable AQ Clients, Delpharm Lille
}
%\end{document} % POUR TEST


